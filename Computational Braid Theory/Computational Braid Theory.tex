\documentclass[english, twocolumn]{article}
\usepackage[T1]{fontenc}
\usepackage[latin9]{inputenc}
\usepackage{babel}
\usepackage{amsthm}
\usepackage{amsfonts}
\usepackage{amsmath}
\usepackage{amssymb}
\usepackage{multicol}
\usepackage{hyperref}
\begin{document}
	
\hypersetup{linktocpage}

\title{Computational Braid Theory}

\author{Jeremy Khawaja}

\onecolumn
\maketitle
	
\begin{abstract}
		A generalization of Turing Machines by using a generalization of topological braids theory.
\end{abstract}
	
\twocolumn
\onecolumn
\tableofcontents{}
\twocolumn
\onecolumn
\begin{quotation}
				
	\begin{center}
				
	\textsl{ "{\footnotesize It sometimes happens that by ignoring a certain amount of data or structure one obtains a simpler, more flexible theory which, almost paradoxically, can give results not readily obtainable in the original setting}"}
				
	\end{center}
					
\end{quotation}

\begin{center}
			
	\textbf{{\small Allen Hatcher}}
			
\end{center}
			
\twocolumn

\newcommand\myeq{\mathrel{\overset{\makebox[0pt]{\mbox{\normalfont\tiny def}}}{=}}}

\newtheorem{defi}{Definition}[section]

\newtheorem{examp}{Example}[section]

\newtheorem{cor}{Corollary}[section]

\newtheorem{theor}{Theorem}[section]

\newtheorem{lem}{Lemma}[section]

\section{Introduction}

\subsection{Turing Machines}

\subsection{Topological Braids}

\section{Computational Braids}

\begin{proof}
	This is a proof.
\end{proof}

\section{Systems Theory}

Since a Computational Braid is just an abstraction for any computer system ...

Systems Theory is about the theory of "Braiding with Braids."

\section{Conclusions}

\end{document}
